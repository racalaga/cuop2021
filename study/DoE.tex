\documentclass{article}
\usepackage{amssymb}
\usepackage{amsmath}
\usepackage{kotex}
\author{Lichen}
\date{\today}
\title{Design of experiment}
\begin{document}
\begin{titlepage}
    \maketitle
\end{titlepage}
    \section{실험설계 개요}
    실험을 통해 향상시키고자 하는 \textbf{\textit{response}} $y$ 를 통제 가능한 변수 $x$, 통제 불가능한 변수 $z$의 조합 그리고 오류항 $\epsilon$으로 나타낼 수 있다.
    \begin{align}
        \begin{bmatrix}
            y_0 \\ y_1 \\ \cdots \\ y_m
        \end{bmatrix} =
        \begin{bmatrix}
             \sum^n_{i=0}\beta_{i0} x_i\\
             \sum^n_{i=0}\beta_{i1} x_i\\
             \cdots\\
             \sum^n_{i=0}\beta_{im} x_i
        \end{bmatrix} 
        +\begin{bmatrix}
            \sum^n_{i=0}\sum^n_{j=0}\beta_{i0} x_i x_j\\
            \sum^n_{i=0}\sum^n_{j=0}\beta_{i0} x_i x_j\\
            \cdots\\
            \sum^n_{i=0}\sum^n_{j=0}\beta_{i0} x_i x_j
        \end{bmatrix}
        +\begin{bmatrix}
            \sum^l_{i=0}\alpha_{i0} z_i+\epsilon_0\\
            \sum^l_{i=1}\alpha_{i1} z_i+\epsilon_1\\
            \cdots\\
            \sum^l_{i=m}\alpha_{im} z_i+\epsilon_m
        \end{bmatrix}
    \end{align}
    
    실제 설계에서는 \textbf{\textit{interation}} 을 나타내는 항 $\sum^n_{i=0}\sum^n_{j=0}\beta_{i} x_i x_j$ 을 구하기 위해 $2^k$ 번의 실험을 진행해야 하기 때문에 모든 경우의 수를 다루는 것이 불가능하다. 

    통제 불가능한 요소가 있기 때문에 $y$는 확률 변수로 표현된다. 
    \begin{align}
        P(y=y_i) = p(y_i)\\
        \int_{\infty}^{-\infty} p(y_i) \,dx = 1
    \end{align}
    
    실험 설계는 다음 세가지 원칙을 따른다.
    \begin{itemize}
        \item \textbf{randomization} : 실험 순서를 무작위로 배치하는 것
        \item \textbf{replication} : 실험을 반복하는 것
        \item \textbf{blocking} : 통제 불가능한 요소가 동일한 조건에서 실험을 진행하여 해당 요소의 영향을 상쇄하는 것
    \end{itemize}

    \section{rescaling}
    실험에 사용하는 요소의 갯수를 축소하는 기법
    일반적인 경우 매개변수를 다음과 같이 축소할 수 있다.
    \begin{align*}
        \frac{dN}{dt} &= \kappa(C_0 -\alpha N)N \\
        \frac{dN}{dt} &= ((\kappa C_0)-(\kappa \alpha)N) = (\widetilde{C_0}-\widetilde{\alpha}N)N, \quad (\widetilde{C_0} = \kappa C_0 \, \widetilde{\alpha} = \kappa\alpha)\\
    \end{align*}
    반응 변수와 시간을 조정하면 하나의 매개변수 방정식으로 바꿀 수 있다.
    \begin{align*}
        &N = \hat{N}N^*, t = \hat{t}t^*\\
        &\frac{dN}{dt} = \kappa(C_0 -\alpha N)N \\
        &\frac{\hat{N}N^*}{\hat{t}t^*} = \kappa(C_0 -\alpha \hat{N}N^*)\hat{N}N^*\\
        &\Rightarrow \frac{dN^*}{dt^*} = (\kappa\hat{t}C_0-\kappa\hat{t}\alpha\hat{N}N^*)N^*\\
        &\Rightarrow \frac{dN^*}{dt} = (1-N^*)N^* , \hat{t} := \frac{1}{\kappa C_0}, \hat{N} := C_0\alpha\\
    \end{align*}

    

    \section{simple comparative experiment}
    한 \textbf{\textit{factor}}에 대해 두 \textbf{\textit{treatment}}가 있는 경우 \textbf{\textit{treatment}}가 \textbf{\textit{response}}에 미치는 영향을 비교.
    \begin{itemize}
        \item T-test
        \item ANOVA
    \end{itemize}
    \section{randomized Block}
    \textbf{\textit{factor}}에 대해 \textbf{\textit{Block}}에서 가능한 모든 \textbf{\textit{treatment}}의 조합을 실험하는 경우 
    블럭 b개 에서 a개의 처리가 가능한 경우

    $y_{ij} = \mu+\tau_i+\beta_j+\epsilon_{ij} \quad \left\{
    \begin{matrix}
    i = 1,2,3,\cdots a\\
    j = 1,2,3,\cdots b
    \end{matrix}\right.$
    \section{Latin square Design}
    2개의 \textbf{\textit{factor}}에 대해 \textbf{\textit{Block}}에서 가능한 모든 \textbf{\textit{treatment}}의 조합을 실험하는 경우 
    각 요소에 대해 한번씩 처리가 나타나도록 배치하여 두가지 변동요소를 제거

    $y_{ijk} = \mu+\aleph_i+\tau_j+\beta_k+\epsilon_{ijk} \quad \left\{
    \begin{matrix}
    i = 1,2,3,\cdots p\\
    j = 1,2,3,\cdots p\\
    k = 1,2,3,\cdots p
    \end{matrix}\right.$
    
    Latin square Design은 Greco-Latin square Design 으로 확장가능

    \section{Factorial Designs}
    \subsection{Randomzied Complete Block Design \textbf{RCBD}}
    \subsection{Two-factor Factorial Design}
    \subsection{General Factorial Design}
    \subsection{$2^k$ Factorial Design}
    \subsection{Two-Level Factional Factorial Design}

    \section{Fitting Regression Model}
    \section{Response Surface Methods and Designs}
    \section{Nested Design}
    \section{Split-plot Design}

\end{document}